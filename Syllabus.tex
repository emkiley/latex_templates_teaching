% Syllabus.tex
% by Erin Kiley <emkiley at mcla dot edu>
% July 03, 2017
% This file is licensed under a Creative Commons Attribution-ShareAlike 3.0 United States License ( https://creativecommons.org/licenses/by-sa/3.0/us/ )
% 
% Sample course syllabus

\documentclass[letterpaper,10pt]{article}

\usepackage{simplemargins,amsmath,amssymb,url,hyperref,longtable,verbatim}
\setallmargins{0.75in}
\setbottommargin{1in}

\providecommand{\head}[1]{\vskip0.8em\noindent\textbf{#1}\vskip0.3em}
\providecommand{\hd}[1]{\noindent\textbf{#1}}

\begin{document}

\begin{center}{\Large{\textbf{MATH 232 (Introduction to Statistics) $\;\cdot\;$ Spring 2017}}}\end{center}

\hd{Instructor:} Dr.\ Kiley \\
\hd{Office:} Bowman Hall, Room 105D \\
\hd{E-mail:} \verb+emkiley at mcla dot edu+ \\
\hd{Telephone:} +1 (413) 662--5144. From on campus, dial 5144. \\
\hd{Office Hours:} MWF 11:00 a.m.--11:50 a.m., T 1:00 p.m.--1:50 p.m., and by appointment.\\
\hd{Optional Extra Help:} T 5:00 p.m.--6:00 p.m., room TBD.
\vskip1em

\hd{Course Number and Section:} Section 01 (CRN 20191, meets MWF 10:00--10:50 in Bowman 206) and Section 02 (CRN 20194, meets MWF 13:00-13:50 in Bowman 208).

\hd{Canvas Page:} Please find our page on MCLA's Canvas system. If you are not automatically granted access to this page, please contact the instructor. You will be notified of important course information through Canvas messaging, and you will be required to submit several of the writing assignments through Canvas, so you \textbf{must} make sure that you have access.

\hd{Required Text:} \emph{Introductory Statistics}, by Barbara Illowsky and Susan Dean. OpenStax College, September 2013. ISBN 978-1-938168-20-8. This is an Open Access book, and can be downloaded for \textbf{free} at: \url{http://cnx.org/content/col11562/latest/}. You may use an electronic copy, or you may order a printed version from various online booksellers.

\hd{Required Text:} \emph{The Cartoon Guide to Statistics}, by Larry Gonick and Woollcott Smith. Newest version is ISBN 978-0062731029, but any edition or printing you can get your hands on will suffice. Available in the MCLA bookstore and at online booksellers. Please obtain your textbook through legal means.

\hd{Required Text:} You will be required to purchase one more book of your choice from a list of titles giving popular treatment of mathematical subjects; you will read this book during the semester, summarize its content, and reflect on its main themes. Please choose \textbf{one} from among the following:
\begin{itemize}
\item \emph{Dataclysm: Love, Sex, Race, and Identity---What Our Online Lives Tell Us about Our Offline Selves}, by Christian Rudder. ISBN 978-0385347396.
\item \emph{Moneyball: The Art of Winning an Unfair Game}, by Michael Lewis, ISBN 978-0393324815.
\item \emph{How Not to Be Wrong: The Power of Mathematical Thinking}, by Jordan Ellenberg, ISBN 978-0143127536.
\item \emph{The Signal and the Noise: Why So Many Predictions Fail---but Some Don't}, by Nate Silver, ISBN 978-0143125082.
\end{itemize}

\head{Goal}

The goal of this course is to examine descriptive statistics, probability, sampling theory, and inferential statistics. With the increasingly larger capacities available for storing data on computers, it has never been more important for global citizens to have literacy in the science of collection, analysis, interpretation, and presentation of data. This is exactly what Statistics is. Descriptive statistics deals with the use of statistical tools to organize and summarize collected information; inferential statistics deals with inferring properties of populations given descriptive statistics taken from a sample.

\head{Classes}

Class time will focus both on delivering course content and on working complementary practical examples. You are expected to maintain your own notebook, and you will be held responsible for knowing the material worked through in each class. In class, you will be asked to form groups, and these groups will change several times over the course of the semester. Within your groups, you will be working through problems and discussing solutions periodically throughout each lecture. Please bring your OpenStax book and your Cartoon Guide to class with you every time we meet.

%\head{Supplemental Instruction}
%
%We are very fortunate, in this course, to have MCLA's Supplemental Instruction (SI) program available to us. Svetlana Morrell (\texttt{sm5653@mcla.edu} ) is an undergraduate student at MCLA who has taken and excelled in MATH 232 before, and we have invited her back to act as a guide and helper for students under the SI program. Svetlana will schedule and conduct two sessions of SI per week, for your benefit. \textbf{All students are strongly encouraged to attend these sessions}, where Svetlana will help you practice and master the course material. It would be a great place to work on your homework, to come with questions about the reading, or to prepare for quizzes, exams, and the project reports. Please feel free to e-mail Svetlana with any questions about the course material, and she will do her best to ensure a positive learning outcome for you in this class.

\head{Attendance Policy}

Students are expected to attend all of their classes and to be aware of course requirements. Whenever possible, students should notify their instructors prior to an absence from class. Students who expect to be absent from classes for three days or longer should contact the Center for Student Success and Engagement for help notifying their instructors. The complete college attendance policy may be located at: \url{http://www.mcla.edu/Academics/registrar/academicpolicies/index }.

\head{Electronic Device Policy}

While it is not forbidden to use electronic devices such as mobile phones, laptops, and tablets in the classroom, it is assumed that if a student is using such a device, it is for legitimate academic purposes (\emph{e.g.}, taking notes, displaying electronic textbooks, using calculator functionality). The instructor reserves the right to request that students display their screens to prove that they are not violating this assumption. Calculators will not be necessary---nor will they be permitted---on quizzes and exams; however, they may be useful for in-class problem solving and/or for homework assignments.


\head{Resources}

If you struggle to understand the course content, there are several opportunities for you to get extra help, both personal/one-on-one, and from external resources. First, the instructor holds an extra help session on Tuesday nights from 5--6 p.m.\ (room TBD), as stated in the heading of the syllabus. These are not additional lectures, and they are optional for students to attend, but they will be focused on solving practical problems. You may also request a tutor from the CSSE office (see the CSSE section of this syllabus). In addition, you are encouraged to regroup with your classmates to work on homework and to study for quizzes and exams (please be mindful of the Academic Dishonesty regulations). You may also stop by office hours to get personal help, and if you come to office hours, it is expected that you will bring your questions and show that you have attempted to solve them. It is essential that you be proactive in your own education.

To this end, each week, under the ``Modules'' section of Canvas, you will also find web links to various videos and tutorials that may also be of use to you. These are not required, but are there for your benefit, especially in case you are getting stuck with the textbook's explanations.

\head{Homework}

Each week, you will be responsible for reading the material assigned on the Canvas page. In mathematics, reading without working through problems is useless---so a small number of exercises will also be assigned \textbf{every class}. The lowest five homework grades will be dropped before the computation of your final course grade. No late assignments will be accepted for credit.

Your homework submissions will receive grades of 0, 1, or 2, corresponding, respectively, to an incomplete or unacceptable submission; a partially complete or partially acceptable submission; and a complete and mostly correct submission that shows effort on your part. Because the number of problems per assignment is small, it is my expectation that your solutions to those problems will be completed with great care and presented professionally. You are encouraged to typeset your solutions using \LaTeX, or to hand-write them very neatly. If the work is not presentable or if it is illegible, you will not receive credit for it. You should take the problems that will be worked in class, in the handouts, and in the textbook as examples of the level of work I expect from you. \textbf{Merely giving the correct answer will receive zero credit.}

You are welcome to discuss homework problems with one another, or to consult external resources such as websites or textbooks, while \emph{thinking through the problems}, but you must \emph{write up your solutions} on your own. Be mindful of your academic integrity.

\begin{comment}
Homework due dates:

\begin{tabular}{rl}
\textbf{Homework 1}: & Monday, 12 Sept 2016 \\
\textbf{Homework 2}: & Friday, 16 Sept 2016 \\
\textbf{Homework 3}: & Monday, 19 Sept 2016 \\
\textbf{Homework 4}: & Friday, 23 Sept 2016 \\
\textbf{Homework 5}: & Monday, 26 Sept 2016 \\
\textbf{Homework 6}: & Friday, 30 Sept 2016 \\
\textbf{Homework 7}: & Monday, 03 Oct 2016 \\
\textbf{Homework 8}: & Friday, 14 Oct 2016 \\
\textbf{Homework 9}: & Monday, 17 Oct 2016 \\
\textbf{Homework 10}: & Friday, 21 Oct 2016 \\
\textbf{Homework 11}: & Monday, 24 Oct 2016 \\
\textbf{Homework 12}: & Monday, 31 Oct 2016 \\
\textbf{Homework 13}: & Friday, 04 Nov 2016 \\
\textbf{Homework 14}: & Monday, 07 Nov 2016 \\
\textbf{Homework 15}: & Friday, 18 Nov 2016 \\
\textbf{Homework 16}: & Friday, 02 Dec 2016 \\
\textbf{Homework 17}: & Monday, 05 Dec 2016 \\
\end{tabular}
\end{comment}

\head{Quizzes}

There will be quizzes given at the beginning of class on most Wednesdays. If you draw a cat on the back of your first quiz, you will get extra credit for reading the syllabus closely. The two lowest quiz grades will be dropped before the computation of your final course grade. There will be no make-up quizzes given. 

\begin{comment}
Quiz dates:
\begin{tabular}{rl}
\textbf{Quiz 1}: & Wednesday, 14 Sept 2016\\
\textbf{Quiz 2}: & Wednesday, 21 Sept 2016\\
\textbf{Quiz 3}: & Wednesday, 28 Sept 2016\\
\textbf{Quiz 4}: & Wednesday, 05 Oct 2016\\
\textbf{Quiz 5}: & Wednesday, 19 Oct 2016\\
\textbf{Quiz 6}: & Wednesday, 26 Oct 2016\\
\textbf{Quiz 7}: & Wednesday, 02 Nov 2016\\
\textbf{Quiz 8}: & Wednesday, 16 Nov 2016\\
\textbf{Quiz 9}: & Wednesday, 30 Nov 2016\\
\textbf{Quiz 10}: & Wednesday, 07 Dec 2016\\
\end{tabular}
\end{comment}




\head{Exams}

There will be three exams, given on the following dates:

\begin{center}
\begin{tabular}{rl}
\textbf{Exam 1}: & Friday, 17 February 2017\\
\textbf{Exam 2}: & Friday, 24 March 2017\\
\textbf{Exam 3}: & Date, time, and location to be announced
\end{tabular}
\end{center}

These exams will not be cumulative, except in the sense that all of mathematics and statistics has been built on a foundation of other mathematics and statistics. If you intend to take your exams in the CSSE office in order to accommodate a learning disability, then it is your responsibility to learn and follow the procedure that they have in place for that.

\pagebreak

\head{Book Project}

This semester, you will be reading a book related to statistics, logic, economics, or data analytics. You will summarize what you have read, and reflect on how it relates to your life by answering a series of prompts that will be given at later dates. Details will be discussed shortly. This project will be worth 12\% of the course grade.

\begin{comment}
\begin{tabular}{rl}
\textbf{First report due}: & Friday, 28 October 2016\\
\textbf{Second report due}: & Monday, 21 November 2016\\
\textbf{Final report due}: & Friday, 09 December 2016
\end{tabular}
\end{comment}


\head{Grading Scheme}

\begin{center}
\begin{tabular}{rrl}
Homework Problems & (drop lowest 5) & 20\% \\
Quizzes & (drop lowest 2) & 15\% \\
Exams & $3\times 16\%$ each & 48\% \\
Book Project and Writing Assignments & & 17\%
\end{tabular}
\end{center}

\head{Final Grades}

You will be assigned a letter grade corresponding to your final course average as follows:

\begin{center}
\begin{tabular}{rll}
$91\%\leq$ &\textbf{A} &\\
$89\%\leq$ &\textbf{A-} & $<91\%$\\
$87\%\leq$ &\textbf{B+} & $<89\%$\\
$80\%\leq$ &\textbf{B} & $<87\%$\\
$78\%\leq$ &\textbf{B-} & $<80\%$\\
$76\%\leq$ &\textbf{C+} & $<78\%$\\
$69\%\leq$ &\textbf{C} & $<76\%$\\
$67\%\leq$ &\textbf{C-} & $<69\%$\\
$62\%\leq$ &\textbf{D} & $<67\%$\\
$60\%\leq$ &\textbf{D-} & $<62\%$\\
&\textbf{F} & $<60$
\end{tabular}
\end{center}



\head{Students with Disabilities}

Any student who believes he or she may need an accommodation based on the impact of a documented disability may be eligible for accommodations that provide equal access to educational programs at MCLA. Students are advised to contact that Disability Resource Office at (413) 662--5318 or stop by CSSE, Eldridge Hall to schedule an appointment. In compliance with the Americans with Disabilities Act (ADA), the Disability Resource Office will work with students to coordinate reasonable accommodations. Students who wish to request accommodations should do so within the first four weeks of the semester. Once accommodations have been determined, the student will provide a copy of his/her accommodation plan to each individual instructor. Students must fulfill all course requirements in order to receive passing grades in their classes, with or without reasonable accommodations. Please note that accommodations cannot be granted retroactively.

\head{Center for Student Success and Engagement}

The Center for Student Success and Engagement (CSSE) offers an integrated array of services and resources to assist your transition to college. Their belief is that every student has the ability to excel academically and be successful, and to this end, they offer a range of peer-advisory programs. If you need academic support, tutoring, or supplemental advising, please stop by their office at the top level of Eldridge Hall.

\head{Counseling Services}

MCLA's Counseling Services offers a range of services including individual and couples counseling, crisis intervention, outreach workshops and educational programming, psychiatric treatment, alcohol and other drug education, consultation to faculty, staff, parents, and students, and off-campus referrals. Group counseling is available as needs arise. Counseling services are confidential and free to all enrolled MCLA students, and it's perfectly normal to ask for help. Counseling Services is located in the MountainOne Student Wellness Center, 2nd Floor, and is open Monday-Friday from 8:30 am to 4:45 pm. Students are seen at Counseling Services by appointment only. To schedule an appointment, please call or drop by the office. Please do not use e-mail to make an appointment.

\pagebreak

\head{Academic Integrity}

A college is a community of students and faculty interested in the search for knowledge and understanding. This requires a commitment to honesty and integrity. Honesty on the part of every college student is integral to higher education at Massachusetts College of Liberal Arts. Acts of dishonesty are not merely a breach of academic honesty but conflict with the work and purpose of the entire College Community. Violations of academic honesty include but are not limited to:
\begin{itemize}
\item Submitting the work of others as one's own
\item Unauthorized communication during or about an examination
\item Use of information (notes, electronic communication, etc.) that is not permitted during exams,
tests, quizzes
\item Obtaining or disseminating unauthorized prior knowledge of examination questions
\item Substitution of another person in an examination
\item Altering College academic records
\item Knowingly submitting false statements, data or results
\item Submission of identical or similar work in more than one course without the approval of the
current instructor
\item Collaborating on material after being directed not to collaborate
\item Forging a signature or false representation of a College official or faculty member or soliciting an
official signature under false pretense
\item Other behavior or activities in completing the requirements of a course that are explicitly
prohibited by an instructor
\item Plagiarism (as defined below)
\end{itemize}

\hd{Plagiarism}: The academic departments of the College have varying requirements for reporting the use of sources, but certain fundamental principles for the acknowledgment of sources apply to all fields and levels of work. The use of source materials of any kind and the preparation of essays or laboratory reports must be fully and properly acknowledged. In papers or laboratory reports, students are expected to acknowledge any expression or idea that is not their own. Students submitting papers are implying that the form and content of the essays or reports, in whole and in part, represent their own work, except where clear and specific acknowledgement is made to other sources. Even if there is no conscious intention to deceive, the failure to make appropriate acknowledgment may constitute plagiarism. Any quotation---even of a phrase---must be placed in quotation marks and the precise source stated in a note or in the text; any material that is paraphrased or summarized and any ideas that are borrowed must be specifically acknowledged. A thorough reordering or rearrangement of an author's text does not release the student from these responsibilities. All sources that have been consulted in the preparation of the essay or report should be listed in the bibliography. Upon an occurrence of alleged academic dishonesty instructors may exercise their discretion in imposing a sanction.

Instructors may also report this sanction to the Registrar or file additional charges against students if they believe that additional sanctions are appropriate. Instructors will notify the Registrar in writing in either or both of the following cases:
\begin{itemize}
\item Any acts of academic dishonesty whenever they have imposed a sanction that is beyond the value of the assignment
\item The instructor requests that the College take further action.
\end{itemize}
The Academic Appeals Committee will hear academic grievances from and about students enrolled
in the undergraduate program. It will also serve as a hearing board for students charged with
academic dishonesty. Further information regarding instructor and student rights and responsibilities and appropriate procedures to be followed in applying this policy may be obtained from the Office of the Dean of Academic Affairs or the Registrar. Additional policies may be found at:

\url{http://www.mcla.edu/Academics/registrar/academicpolicies/index}

Policies with approval dates and text (focused on curriculum) can be found by clicking on
``Approved Courses and Policies'' at the left of the Campus Collaboration page, then clicking on ``Undergraduate Policies''.

\end{document}

\pagebreak

\head{Tentative Schedule}

Please find below the tentative schedule for the class. You are expected to have finished reading the indicated sections in the OpenStax (`OS') and Cartoon Guide (`CG') texts before the lecture begins.

\begin{center}
\begin{longtable}{lp{3cm}p{9cm}p{2.5cm}}
Date & Deliverables & Topic & Reading (do \textbf{before} lecture)\\
\hline
\endhead
\hline
\multicolumn{4}{r}{Continued on next page\ldots}\\
\endfoot
\hline
\hline
\endlastfoot
W 18 Jan & & \emph{Welcome; Measures of Central Tendency; Mean, Median, and Mode for Ungrouped Data} & pp.\ 46--48 \\
F 20 Jan & Homework & \emph{Measures of Dispersion; Range, Variance, and Standard Deviation for Ungrouped Data} & pp.\ 48-51 \\
M 23 Jan & Homework 1 due. & \emph{Measures of Central Tendency and Dispersion for Ungrouped Data; Chebyshev's Theorem; Empirical Rule; Coefficient of Variation; Z Scores} & pp.\ 51--54 \\
W 25 Jan & Quiz 1. & \emph{Measures of Position: Percentiles, Deciles, Quartiles; Interquartile Range; Box-and-Whisker Plot} & pp.\ 54--56\\
F 27 Jan & Add/drop ends. & \emph{Experiment, Outcomes, Sample Space; Tree Diagrams and the Counting Rule; Events, Simple Events, and Compound Events} & pp.\ 71--73\\
M 30 Jan & Homework 3 due. & \emph{Probability; Classical, Relative Frequency and Subjective Probability; Marginal and Conditional Probabilities; Mutually Exclusive Events} & pp.\ 73--78\\
W 01 Feb & Quiz 2. & \emph{Dependent and Independent Events; Complementary Events; Multiplication Rule for the Intersection of Events; Addition Rule for the Union of Events} & pp.\ 78--81\\
F 03 Feb & Homework 4 due. & \emph{Bayes' Theorem; Permutations and Combinations; Using Permutations and Combinations to Solve Probability Problems} & pp.\ 81--84\\
M 06 Feb & Homework 5 due. & \emph{Random Variables; Discrete Random Variables; Continuous Random Variables; Probability Distribution; Mean of a Discrete Random Variable; Standard Deviation of a Discrete Random Variable} & pp.\ 98--102\\
W 08 Feb & Quiz 3. & \emph{Binomial Random Variables; Binomial Probability Formula; Tables of the Binomial Distribution; Mean and Standard Deviation of a Binomial Random Variable} & pp.\ 102--106\\
F 10 Feb & Homework 6 due. & \emph{Poisson Random Variable; Poisson Probability Formula; Hypergeometric Random Variable; Hypergeometric Probability Formula} & pp.\ 106--110\\
M 13 Feb & Homework 7 due. & \emph{Uniform Probability Distribution; Mean and Standard Deviation for the Uniform Probability Distribution; Normal Probability Distribution} & pp.\ 124--128\\
W 15 Feb & Quiz 4. & \emph{Review for Exam 1} & \\
F 17 Feb & {Exam 1} & & \\
M 20 Feb & & \emph{Standard Normal Distribution; Standardizing a Normal Distribution} & pp.\ 128--132\\
W 22 Feb & Homework 8 due. & \emph{Applications of the Normal Distribution; Determining the z and x Values When and Area under the Normal Curve is Known} & pp.\ 132--137\\
F 24 Feb & Homework 9 due. & \emph{Normal Approximation to the Binomial Distribution; Exponential Probability Distribution; Probabilities for the Exponential Probability Distribution} & pp.\ 137--141\\
M 27 Feb & Quiz 5. & \emph{Simple Random Sampling; Using Random Number Tables; Using the Computer to Obtain a Simple Random Sample} & pp.\ 152--154\\
W 01 Mar & Homework 10 due. & \emph{Systematic Random Sampling; Cluster Sampling; Stratified Sampling; Sampling Distribution of the Sampling Mean} & pp.\ 154--156\\
F 03 Mar & Homework 11 due. & \emph{Mean and Standard Deviation of the Sample Mean; Shape of the Sampling Distribution of the Sample Mean and the Central Limit Theorem; Applications of the Sample Mean} & pp.\ 156--160\\
M 06 Mar & Quiz 6. & \emph{Sampling Distribution of the Sample Proportion; Mean and Standard Deviation of the Sample Proportion; Shape of the Sampling Distribution of the Sample Proportion and the Central Limit Theorem; Applications of the Sampling Distribution of the Sample Proportion} & pp.\ 160--164\\
W 08 Mar & Homework 12 due. & \emph{The t Distribution; Confidence Interval for the Population Mean: Small Samples} & pp.\ 182--187\\
F 10 Mar & Quiz 7. & \emph{Confidence Interval for the Population Proportion: Large Samples; Determining the Sample Size for the Estimation of the Population Mean; Determining the Sample Size for the Estimation of the Population Proportion} & pp.\ 187--190\\
M 20 Mar & Homework 13 due. & \emph{Null Hypothesis and Alternative Hypothesis; Test Statistic, Critical Values, Rejection and Nonrejection Regions} & pp.\ 200--202\\
W 22 Mar & End of Withdraw & \emph{Review for Exam 2} & \\
F 24 Mar & {Exam 2} & & \\
M 27 Mar & & \emph{Type I and Type II Errors} & pp.\ 202--207\\
W 29 Mar & Quiz 8. & \emph{Hypothesis Tests about a Population Mean: Large Samples; Calculating Type II Errors} & pp.\ 207--212\\
F 31 Mar & Homework 15 due. & \emph{P Values; Hypothesis Tests about a Population Mean: Small Samples; Hypothesis Tests about a Population Proportion: Large Samples} & pp.\ 212--219\\
M 03 Apr & Project Report 2 Due. & \emph{F Distribution; F Table; Logic Behind a One-Way ANOVA} & pp.\ 299--304\\
W 05 Apr & & \emph{Sum of Squares, Mean Squares, and Degrees of Freedom for a One-Way ANOVA; Sampling Distribution for the One-Way ANOVA Test Statistic; Building One-Way ANOVA Tables and Testing the Equality of Means} & pp.\ 304--311\\
F 07 Apr & Quiz 9. & \emph{Logic Behind a Two-Way ANOVA; Sum of Squares, Mean Squares, and Degrees of Freedom for a Two-Way ANOVA} & pp.\ 311--315\\
M 10 Apr & Homework 16 due. & \emph{Building Two-Way ANOVA Tables; Sampling Distributions for the Two-Way ANOVA; Testing Hypothesis Concerning Main Effects and Interaction} & pp.\ 315--320\\
W 12 Apr & Homework 17 due. & \emph{Straight Lines; Linear Regression Model} & pp.\ 339--342\\
F 14 Apr & Quiz 10. & \emph{Least-Squares Line; Error Sum of Squares; Standard Deviation of Errors; Total Sum of Squares; Regression of Sum of Squares} & pp.\ 342--348\\
W 19 Apr & Final Project Report Due. & \emph{Mean, Standard Deviation, and Sampling Distribution of the Slope of the Estimated Regression Equation; Inferences Concerning the Slope of the Population Regression Line} & pp.\ 348--350 \\
F 21 Apr & & \emph{Estimation and Prediction in Linear Regression; Linear Correlation Coefficient; Inference Concerning the Population Correlation Coefficient} & pp.\ 350--354\\
M 24 Apr & & \emph{} & pp.\ \\
W 26 Apr & & \emph{} & pp.\ \\
F 28 Apr & & \emph{} & pp.\ \\
M 01 May & & \emph{} & pp.\ 
\end{longtable}
\end{center}



\end{document}